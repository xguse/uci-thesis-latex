

\chapter{Introduction}
\todo[inline]{replace CITEME tags in Introduction}
% \section{Purpose}
% 
% Introduce core concepts relating to my project such as:

% \begin{enumerate}
% \item \chkBoxX Importance of Dengue and Malaria
% \item \chkBoxX Transmission cycle dependent on Mosquitoes
% \item \chkBox Limited effectiveness (medical/finacial) of treatment of the diseases in rural poor human communities
% \item \chkBoxX Vector Control Strategies
% \item \chkBoxX How Transgenesis modifies the vector control landscape
% \item \chkBox Elements of Successful Transgenesic Vector Control
% \item \chkBoxX Introduction to control of transcription and imoprtance of \glspl{CRE}/\glspl{CRM}
% \end{enumerate}




\section{Dengue fever and malaria are mosquito borne diseases}

There are many mosquito-borne diseases that afflict the world's population.
Dengue fever and malaria are two important such diseases resulting in major health and productivity deficits in many developing nations worldwide \cite{Gubler2012,Sachs2002}.
The causative agents of these two diseases are a suite of four species of RNA viruses (Dengue Virus 1-4) and a group of five species of protist from the genus \textit{Plasmodium} (\Plf, \Pv, \Po, \Pm, and \Pk).
However, in any single geographic location each disease is transmitted primarily by a single species of mosquito (Table \ref{tab:species-disease-region}). 

\input{tables/species-disease-region}

The fact that both of these diseases are caused by multiple pathogen species complicates efforts to prevent human disease: particularly when addressed from the perspective of providing therapeutic or prophylactic care directly to human patients.
Dengue vaccine development is complicated by the fact that patients with an established immunity to one serotype are more likely to develop a more sever form of the disease if they become infected with a second serotype because of a phenomenon known as antibody-dependent enhancement \cite{Flipse2013}.
While different species of the malaria parasite present slightly different versions of the disease and require specific treatment considerations \cite{Prevention}.  

\section{Hematophagy (bloodfeeding) is an essential part of mosquito-borne disease}

Another very important aspect of both of these diseases is that people do not become infected through interaction with other people, and mosquitoes do not become infected through interaction with other mosquitoes.
The transmission cycles transmission of both diseases require that the pathogen be passed from human to mosquito or from mosquito to human.
This pattern has important consequences that can be exploited to reduce the number of people who will become infected with dengue fever or malaria.

\subsection{Transmission cycle}

\textbf{In Brief:}
\begin{enumerate}
\item A female mosquito bites an infected person.
\item The pathogen is taken into the mosquito's midgut along with bloodmeal.
\item The pathogen must escape the midgut and gain access to the mosquito's circulatory system (the animal is now \textbf{infected}).
\item The pathogen must gain access to the mosquito's salivary glands (the animal is now \textbf{infectious}).
\item The infectious female mosquito bites an uninfected human and the pathogens in her saliva are introduced to the person's body as she feeds.
\end{enumerate}


\begin{figure}[h]
\centering 
\includegraphics[width=.6\linewidth]{figures/figs/mosqXsection.pdf}
\caption[Cross section of a female mosquito]{\sf \textbf{Cross section of a female mosquito with tissues important to the transmission cycle highlighted.} \\  \CITEME}\label{fig:mosq-xsection}
\end{figure}


While the specifics of how viruses and protists live and reproduce in mosquitoes and humans are quite different, the transmission cycles of dengue and malaria are similar in some critical events.
The transmission cycle of either pathogen type can be encapsulated into 5 fundamental events.
First, a female mosquito feeds on an infected human's blood taking up the pathogen as well (Figure \ref{fig:mosq-xsection}).
The bloodmeal is digested in the mosquito's midgut, and escape from the midgut is the first crucial step for the survival of the pathogen inside the mosquito.
If it is trapped in the midgut, it will eventually be passed as waste after the bloodmeal is digested.
If it manages to escape to the \gls{hemolymph}, the pathogen must survive long enough to gain access to the mosquito's salivary glands.
If it is unable to infiltrate the salivary glands, then the mosquito, while itself \textbf{infected}, is not \textbf{infectious} to other humans.
For this, the pathogen must be injected into the bloodstream of the next human along with the contents of the mosquito's saliva.

\section{Public health: \emph{multiple targets}}

Because the mode of transmission essentially links mosquito to humans or humans to mosquitoes, the cycle will collapse and a local area would eventually be cleared of the pathogen if either of these links are broken, provided that the intervention is maintained.
People do not infect other people.
This model suggests two primary targets for public health interventions aiming to reduce the \gls{incidence}
(followed by \gls{prevalence}) of infection in a population:

\begin{enumerate}
\item Prevent mosquitoes from infecting humans.
\item Prevent humans from infecting mosquitoes.
\end{enumerate}

However, many interventions can not be clearly divided into addressing purely the human or vector side of the cycle.
For example, efforts to reduce the number of bites that humans receive from mosquitoes affects both the probabilities of the vector \emph{and} the human becoming infected.
This is one reason that vector control is frequently part of prevention strategies.
For the purposes of this document, I will define human-based interventions to be those that are directly administered to a human body: things like prophylactic- or therapeutic-drugs and vaccines.
In this respect, the long-term goals of public health are more focused on prevention than the treatment of acute cases.
Of course, sick people need to be treated. Additionally, identifying and curing people who are infected also prevents mosquitoes from acquiring the pathogen.
However, prevention is preferred because it is more cost-effective.


\subsection{Addressing the human side of transmission}

Because of the definition of human-based interventions above, there are relatively few effective options in this category for malaria and dengue fever.
Normally, this section would include vaccines and swift, effective patient identification and treatment to clear the infection.
However, at this time these types of interventions are still in development phases or do not yet exist in a cost-effective form applicable to the isolated and impoverished areas that are most affected \cite{Graves2006,DelAngel2013,Amin2005}.

\subsection[Addressing the vector side of the cycle (conventionally)]{Non-transgenic options for addressing the vector side of the cycle}

Medical options are scarce generally in the regions of the world where these diseases are most common.
The alternative is to focus attention toward controlling access of the vector populations to human contact.
This can include removal of nearby mosquito breeding sites (usually various forms of standing water), spraying of insecticides, introduction of biological predators, and/or bed nets, etc.
One fairly recently developed approach that has shown great promise is \gls{SIT}.



\paragraph*{Sterile insect technique:}
\gls{SIT} exploits an aspect of the reproductive behavior of some insects. 
In \Aa, the first mating event provides the only genetic material that the female will use to fertilize her eggs for the rest of her life \cite{Craig1967}.
Subsequent mating events, if they occur, contribute little to no genetic material to the females progeny, \textbf{even if the first event involves a sterile male}.
This means that release of inundative numbers of sterilized males into a native population will effectively sterilize significant proportions of wild females.
This has dramatic effects on the local population.
Negative population growth results from the inability of each generation to fully replace itself, and the effect is compounded for each subsequent generation as long as the program is maintained.
Perhaps most famously, \gls{SIT} was used successfully to eradicate the screwworm from the southern United States in the 1950s \cite{Bushland1955}.


\subsection[Transgenesis provides new vector control strategy]{Transgenic techniques provide a new strategy for controlling disease transmission}

Most conventional vector control strategies involve what might be termed vector \gls{population-reduction}.
In addition to \gls{population-reduction}, transgenic techniques provide a novel vector control strategy that has considerable implications to the sustainability of the vector control aspect of public health interventions for dengue and malaria.
In contrast to \gls{population-reduction}, this novel strategy could be conceptualized as vector \gls{population-modification}.
The goal is not to reduce or eliminate contact between humans and vectors but to drive a transmission-refractory trait through the vector population that renders the vector unable to become infectious, converting it into a non-vector.

\section{Elements of a successful transgenic vector control product}

\begin{enumerate}
  \item the ability to introduce transgenes into the vector genome in a heritable way
  \item the discovery or construction of effector genes
  \item the ability to control when and where the effector genes are expressed
%   \item the ability of your transgenic vectors to have swift, dramatic, and clinically meaningful effects on the native vector population (gene drive)
  \item the ability of transgenic vectors to spread anti-pathogen genes to near-fixation on a public health timescale
\end{enumerate}



\subsection{Introducing heritable transgenes into mosquito genomes}

The capacity to make transgenic vectors is essential to the design of transgenic methods for vector control.
For mosquitoes, this is more labor intensive than for other fly species like \Dmel.
However, it has been accomplished, and protocols to reliably introduce transgenes into many mosquito species now exist \cite{Adelman2007}.
Unfortunately, the relative difficulty is high.
It takes 2-3 months to establish a stable line.
Thus many experimental approaches that are straightforward or even routine in \Dm\ or other species remain impractical for mosquitoes.

The method of introducing a \gls{transgene} into a mosquito genome generally relies on re-purposing the activities of \glspl{transposon} (\glspl{transposable-element}).
One class of these mobile genetic elements has the ability to cut themselves out of one part of the DNA and re-insert themselves into another.
It is possible to engineer them to insert selected transgenes into the mosquito genome instead of the genes that they usually contain.


Until recently, most if not all transposons used for transgenesis in mosquitoes inserted the transgenes into more-or-less random locations.
The number of locations and number of copies in any single genome was unpredictable \cite{Adelman2004,Sethuraman2007}.
This complicates the characterization of the effects of a transgene.
However, with the successful implementation of the \gls{phic31} site-specific integration system in \Aa\ and \As\ \cite{Thorpe1998,Nimmo2006,Isaacs2012}, we can now reliably target specific, pre-characterized regions of these genomes for transgene integration of a single copy.

Heritability is achieved through specifically targeting the \gls{germline} cells of mosquito embryos for transgenesis.
In flies like mosquitoes, the cells that will become the germline always position themselves at one end (or pole) of the embryo (Figure \ref{fig:pole-cells}).
The transgenesis cocktail containing the transgene construct and a source of transposase is injected into the embryos where the pole cells are located.
This means that the first generation of mosquitoes that may contain the transgene in every cell of their body is \textbf{not} the injected generation but the progeny of the injected mosquitoes.


\begin{figure}[h]
\centering \sffamily

\includegraphics[width=.8\textwidth]{figures/figs/oskar_expression_AaAg.png}

\caption[\textit{oskar} gene expression in the vector mosquitoes, \Ang\ and \Aea]{\sf \textbf{\textit{oskar} gene expression in the vector mosquitoes, \Ang\ and \Aea:} \\
 Hybridizations \textit{in situ} of \textit{Anga osk} and \textit{Aeae osk} RNA probes to whole-mount \Ag\ and \Aa\ embryos. Embryos are orientated with anterior on the left. \Ag\ embryos (A,B) shown have reached cellular blastoderm with pole cells indicated by an arrow. \Aa\ embryos (C,D) shown are in the preblastoderm stage. Strong staining is exclusive to the posterior pole cells (A) and posterior end (C) (arrows), when compared with embryos of the same stage, determined by DAPI staining (not shown), probed with \textit{Anga} and \textit{Aeae osk} sense RNA probes (B,D). Bar = 50 µm.\\
 
 Excerpted from \cite{Juhn2006}}\label{fig:pole-cells}
\end{figure}

\subsection{Effector genes}

In addition to the ability to integrate transgene constructs into the vector genome, it is necessary to design constructs that contain transgenes that will effect a useful response in the vector.
Two broad strategies are commonly applied when designing an anti-transmission phenotype: \gls{population-reduction} and \gls{population-modification}.

\paragraph*{Vector population reduction:}

The goal of vector \gls{population-reduction} is the same as conventional vector control modalities: reduce the local population of a disease vector with the goal of reducing the probability that an infectious individual will have an opportunity to participate in a transmission event.


A transgenic tactic of the \gls{population-reduction} strategy is the \gls{RIDL} method.
An example is the OX3604C strain of \Aa\ which uses a poison transgene to kill female adults that express the transgene unless the female is provided an ``antidote'' through its water supply \cite{WisedeValdez2011,Bargielowski2012,Facchinelli2013}.

Releasing sufficient numbers of these mosquitoes is anticipated to affect the local mosquito species population in a way that is analogous to \gls{SIT}.
Trials of \gls{RIDL} strains are currently underway in Brazil and the Cayman Islands.

\paragraph*{Vector population modification:}

Vector \gls{population-modification} is a novel strategy for vector control that exists primarily within the context of genetically altered vectors.
It promises to be the most long-lasting vector-focused intervention.
Unlike all vector \gls{population-reduction} tactics, there is a predicted point in a conversion intervention (the anti-transmission trait works and is present in local populations at near 100\%) when the human management of the intervention can be ceased, yet it continues to function.
For \gls{population-reduction} to approach this result, the vector species must not simply be eliminated from the local area, but approach elimination on a continental scale, or more realistically achieve global eradication.

The reason is that these mosquitoes (especially \Aa) can and do travel long distances if transported by humans as eggs or larvae in the backs of trucks (between villages) or in the pools of water collected in super-tankers (transcontinentally).
So a local village is only free of the vectors until more migrate into the area.
But in a conversion scenario, those migrants mate with the local vector population, and their offspring are assimilated into transmission-deficient mosquitoes.
The protection of the local village can be preserved \emph{even if some surrounding villages fail to maintain control of their mosquitoes}.

An example of a transgenic tactic for \gls{population-modification} of \Aa\ into a transmission-deficient phenotype involves an effector gene that produces a double-stranded RNA molecule that contains a portion of the dengue virus genome \cite{Franz2006,Mathur2010}.
Because most animal cells have a system that detects and degrades double-stranded RNA\footnote{Double-stranded RNA is produced as a viral replication intermediate in the cell.}
in a sequence specific manner, this primes the mosquito cells' antiviral response to specifically attack the dengue virus if the effector gene is expressed in the cell before it gets infected.

\subsection{Controlling when and where the effector genes are expressed}

Even with an effector gene that clears 100\% of the pathogen 100\% of the time, the intervention will not be successful in limiting transmission if the transgene is not turned on at the right time and place.
For example, if the transgene is most effective when the pathogen is in the midgut, it is not useful if the gene is only expressed in the antennae.

The region of DNA directly upstream of the coding sequence of a gene is a major determinant of the pattern of expression and is referred to as the promoter.
It influences when the gene will be turned on (e.g., directly following the ingestion of a bloodmeal), in which tissue type (e.g., the midgut), and in which sex (e.g., females).
Promoters influence the transcription of genes through the binding of special proteins called transcription factors that recognize specific DNA sequences called \glspl{CRE}.
\glspl{CRE} are sometimes organized into modules that represent groups of binding sites for a specific combination of transcription factors.
These modules are \glspl{CRM} (Figure \ref{fig:cre-crm-intro-a}) \cite{Davidson2010}.
Once a set of transcription factors are bound to the promoter they recruit additional proteins necessary for the gene to be expressed or silenced depending on the context of the cellular environment (Figure \ref{fig:cre-crm-intro-b}).



\begin{figure}[h]

\begin{subfigure}[b]{.5\linewidth}
\centering 
\includegraphics[width=0.9\linewidth]{figures/figs/CREvCRM.pdf}
\caption{}\label{fig:cre-crm-intro-a}
\end{subfigure}%
\hfil
\begin{subfigure}[b]{.5\linewidth}
\centering
\includegraphics[width=0.9\linewidth]{figures/figs/elementsOfTxReg.jpeg}
\caption{}\label{fig:cre-crm-intro-b}
\end{subfigure}

\caption[Elements of transcriptional regulation]{\sf \textbf{Elements of transcriptional regulation} \\
(A) The relationship between \glspl{CRE} (colored rectangles) and \glspl{CRM}: the colored rectangles represent position specific scoring matrix information (top) that defines an individual \gls{CRE}.
(B) \glspl{TF} bind to specific sites (\glspl{TFBS}) that are either proximal or distal to a transcription start site. Sets of \glspl{TF} can operate in functional \glspl{CRM} to achieve specific regulatory properties. Interactions between bound \glspl{TF} and cofactors stabilize the transcription-initiation machinery to enable gene expression. The regulation that is conferred by sequence-specific binding \glspl{TF} is highly dependent on the three-dimensional structure of chromatin.\\

Panel B and corresponding legend text adapted from \citet{Wasserman2004}.
}\label{fig:cre-crm-intro}
\end{figure}

A typical strategy to obtain regulatory sequences suited to drive a transgene following the expression characteristics used as an example above (expressed directly following a bloodmeal, in the midgut) would be to identify genes in the mosquito that already have a similar expression pattern to the ideal.
The promoter from that gene would then be used to direct expression of the transgene from the final integration construct.

This is a common approach to engineer the expression patterns of real transgenes in mosquitoes.
In \citet{Moreira2000}, the promoter sequences of a gene encoding a carboxypeptidase enzyme (normally expressed in the midgut after a meal) from \Aa\ and \Ag\ were inserted in front of a reporter gene.
This resulted in abundant expression of \gls{luciferase} mRNA and protein in the midguts of transgenic mosquitoes in four of six AeCP lines.
Expression of the reporter gene was gut-specific and peaked about 24h \gls{PBM}.


\subsection{Spreading anti-pathogen traits to near-fixation on a public health timescale}

In many ways, this is the most difficult part of the system.
In order for the transgene to have its self-sustaining properties as well as achieve effective anti-transmission results, it must spread through the native mosquito population to the point that the percent of individuals possessing the gene approaches 100\%.

Making an assumption that the transgene carries a negligible fitness cost\footnote{By fitness cost/advantage here, I mean that the transgene
    causes the mosquitoes that inherit it to be either less or more
    successful at producing offspring, respectively.},
or even a slight fitness advantage, achieving near 100\% conversion would take far too long.
Funding for efforts like this in poor nations likely would not last long enough to achieve success.
To enable the \gls{population-modification} strategies to work, genetic ``tricks'' must be exploited that cause the gene to spread through a mosquito population much faster than would happen naturally.
Efforts to discover or design these \gls{gene-drive} systems are an on going and active area in this field.
An example of such a method is the Medea system successfully engineered in \Dm\ \cite{Ward2011}.


\textbf{In Brief:}
\begin{enumerate}
 \item The construct consists of a \text{death gene} encoding a toxin under control of a maternal effect promoter and a \textbf{rescue gene} encoding an antidote under control of an early embryonic promoter.
 \item The \textbf{death gene} is produced by the mother and deposited into 100\% of her developing embryos: \textit{even those that did not inherit the transgene due to mating with a wild type male.}
 \item However, since the \textbf{rescue gene} is produced by only those embryos that inherited the transgene construct, only those embryos will survive.
 \item The female has on average half as many progeny survive as usual if she mated with a wild type male, but 100\% of those that survive inherit the transgene.
 \item Over time this results in the fixation or near fixation of the transgene in the population.
\end{enumerate}


\section{Evolutionary relationship between mosquitoes and hematophagy}
There is debate on when and how \gls{hematophagy} arose in mosquitoes \cite{Rai1999a}; however, it is hypothesized that the trait is \gls{monophyletic} \cite{Borkent2004,Calvo2006}.
This suggests that the use of evolutionary information and relationships may be used to improve the discovery of functionally related gene-sets.
Gene-sets based on transcriptional profiles that are conserved between mosquito species due to the shared ancestry of orthologous genes also have implications for the design of transgenic vector control systems.
Promoter elements derived from these types of relationships are more likely to have similar effects across multiple species and strains of mosquitoes which may reduce the time and development cost to develop transgenic strains in new geographic locations.
The hypothesis of a common ancestry of the bloodfeeding trait has inspired the approach used in this project.



% \briefConclude{
% \begin{itemize}
% \item there are many but we focus on Dengue Fever and Malaria
% \item causative agents are RNA viruses and Plasmodium protists, respectively
% \item people don't give people these two diseases
% \item mosquitoes generally don't give mosquitoes these diseases
% \end{itemize}
% }
% \synopsis{
% \begin{itemize}
% \item The specifics of the transmission cycle of these diseases provides multiple targets for public health interventions
% \item focus on the human
% \item focus on the vector
% \end{itemize}
% }