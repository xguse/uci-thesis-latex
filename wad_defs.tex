%%%%%%       Define Colors
\definecolor{grey}{RGB}{152,152,152}
\definecolor{brick}{RGB}{160,0,0}
% % % Math Definitions: 
\DeclareMathOperator*{\median}{median} 
\providecommand{\e}[1]{\ensuremath{\times 10^{#1}}}

% commands for common volumes and things that are annoying to type
\newcommand{\phic}{$\phi$C31}
\newcommand{\nfkb}{NF$\kappa$B}
\newcommand{\mul}{$\mu$l}
%\newcommand{\ul}{$\mu$l}
\newcommand{\ug}{$\mu$g}
\newcommand{\uM}{$\mu$M}
\newcommand{\degree}{\ensuremath{^\circ}}
\newcommand{\C}{\degree C}
\newcommand\water{H$_2$O}
\newcommand\coo{CO$_2$}
\newcommand\chlf{CHCl$_3$}
\newcommand\chkBox{ {\Square} }
\newcommand\chkBoxX{ {\CheckedBox} }




% miniSolns are little on-the-spot mixes rather than stock solutions
% \newcommand\miniSoln[1]{%
%         %\bigskip
%         \begin{center}\sffamily%
%                 \ovalbox{\parbox[l]{5in}{\vskip1ex \hskip1ex \textbf{Solution:} \textcolor{black}{#1}}}%
%         \end{center}%
%         %\bigskip%
% }


\newcommand\miniSoln[1]{%
        %\bigskip
        \begin{center}\sffamily%
                \ovalbox{
					\begin{minipage}[c]{.8\linewidth}
					\vskip1ex \hskip1ex \textbf{Solution:} \textcolor{black}{#1}
					\end{minipage}
				}%
        \end{center}%
        %\bigskip%
}

\newcommand\synopsis[1]{%
        %\bigskip
        \begin{center}%
                \fbox{
					\begin{minipage}[c]{.95\linewidth}
					\vskip1ex \hskip1ex \textbf{Synopsis: \\} \textcolor{black}{#1}
					\end{minipage}
				}%
        \end{center}%
        %\bigskip%
}

\newcommand\topicBox[2]{%
        %\bigskip
        \begin{center}%
                \fbox{
					\begin{minipage}[c]{.95\linewidth}
					\vskip1ex \hskip1ex \textbf{#1 \\} \textcolor{black}{#2}
					\end{minipage}
				}%
        \end{center}%
        %\bigskip%
}

% bioCheats are hacks that might work as a last resort to fix a failed experiment
\newcommand\bioCheat[1]{%
        %\bigskip
        \begin{center}\sffamily% 
                \ovalbox{\parbox[l]{5in}{\textbf{Bio-cheats:} \textcolor{green}{#1}}}%
        \end{center}%
        %\bigskip%
}

% bioTip are little suggestions for getting slightly better results or optimization
\newcommand\bioTip[1]{%
        %\bigskip
        \begin{center}\sffamily%
                \ovalbox{\parbox[l]{5in}{\textbf{BioTip:} \textcolor{green}{#1}}}%
        \end{center}%
        %\bigskip%
}

% valuable lessons are little problems that you finally figure out how to solve
\newcommand\valuableLesson[1]{%
	%\bigskip
	\begin{center}\sffamily%
		\ovalbox{\parbox[l]{5in}{\textbf{Valuable Lesson:} \textcolor{blue}{#1}}}%
	\end{center}%
	%\bigskip%
}

% gotchas are things that could cause your experiment to fail if you aren't careful
\newcommand\gotcha[1]{%
	%\bigskip
	\begin{center}\sffamily%
		\ovalbox{\parbox[l]{5in}{\textbf{Gotchas:} \textcolor{red}{#1}}}%
	\end{center}%
	%\bigskip%
}

% this is how I link to raw data
% you have to update this url to whereever you put your own data

% brief conclusions sum up a section
\newcommand\briefConclude[1]{\paragraph{Brief Conclusions:} #1}

% brief updates are added later after I learn something that might be relevant to a previous section
%\newcommand\briefUpdate[2]{\paragraph{Brief Update \emph{#1}:} \textcolor{magenta}{#2}}
\newcommand\briefUpdate[2]{%
	%\bigskip
	\begin{center}%
	\begin{tikzpicture}
 		\node [fill=shade,rounded corners=7pt]
		{ \parbox[l]{6in}{\bsf{Brief Update \emph{#1}:} \textcolor{magenta}{\bsf{#2}}} };
	\end{tikzpicture}
	\end{center}%
	%\bigskip%
}

% to dos are temporary reminders of stuff I'd like to do;  I usually try to remove 
% them after I've done the stuff.
\newcommand\toDo[1]{\paragraph{\textcolor{green}{To Do!!!}}\textcolor{red}{#1}}

% format a file name
\newcommand\fName[1]{\texttt{\textcolor{green}{#1}}}

% format a simple command
\newcommand\cmd[1]{\texttt{\textcolor{brick}{#1}}}

% format a bold sans serif
\newcommand\bsf[1]{\textbf{\textsf{#1}}}

% format a sans serif
\newcommand\nsf[1]{\textsf{#1}}

% Alert text Style
\newcommand\alert[1]{\textcolor{magenta}{#1}}

% rArrow short
\newcommand\rArw{$\Rightarrow$}

% et al
\newcommand\etal{\textit{et al.}}

%paraBreak
\newcommand{\pb}{\vskip2.5ex}

\newcommand{\CITEME}{\alert{CITEME}}

% BLIND TEXT DEFINITIONS
\usepackage{lipsum}
\newcommand{\dummytext[1]}{\alert{\lipsum[#1]}}