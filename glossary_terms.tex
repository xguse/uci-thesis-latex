
% \newglossaryentry{TAG}{name={NAME},
% description={xxxxxxxxx}}

\newglossaryentry{luciferase}{name={luciferase},
description={a class of oxidative enzymes that produces bioluminescence}}

\newglossaryentry{Swiss-Prot}{name={Swiss-Prot},
description={the manually annotated and reviewed section of the UniProt Knowledgebase \url{http://www.uniprot.org/}}}

\newglossaryentry{Pfam}{name={Pfam},
description={\textit{from the Pfam website:} The Pfam database is a large collection of protein families, each represented by multiple sequence alignments and \glspl{HMM} \url{http://pfam.sanger.ac.uk/} }}


\newglossaryentry{bowtie}{name={bowtie},
description={ is an ultrafast and memory-efficient tool for aligning sequencing reads to long reference sequences. It is particularly good at aligning reads of about 50 up to 100s or 1,000s of characters, and particularly good at aligning to relatively long (e.g. mammalian) genomes (\url{http://bowtie-bio.sourceforge.net/bowtie2/index.shtml}) }}

\newglossaryentry{tophat}{name={tophat},
description={ is a fast splice junction mapper for RNA-Seq reads. It aligns RNA-Seq reads to mammalian-sized genomes using the ultra high-throughput short read aligner Bowtie, and then analyzes the mapping results to identify splice junctions between exons (http://tophat.cbcb.umd.edu/)}}

\newglossaryentry{cufflinks}{name={cufflinks},
description={ both the name of a suite of programs and a member of that suite.  Members include cufflinks, cuffmerge, and cuffdiff.  As a whole, it assembles transcripts, estimates their abundances, and tests for differential expression and regulation in RNA-Seq samples. It accepts aligned RNA-Seq reads and assembles the alignments into a parsimonious set of transcripts. Cufflinks then estimates the relative abundances of these transcripts based on how many reads support each one, taking into account biases in library preparation protocols.  (\url{http://cufflinks.cbcb.umd.edu/})}}

\newglossaryentry{cuffmerge}{name={cuffmerge},
description={a member of the suite of programs named ``cufflinks''. Responsible for merging condition specific transcriptomes into a single experiment-wide transcriptome used to estimate the expression coverage and differential expression at later points in the process.}}

\newglossaryentry{cuffdiff}{name={cuffdiff},
description={a member of the suite of programs named ``cufflinks''. Responsible for estimating the expression coverage and differential expression of transcripts contained in the single experiment-wide transcriptome generated by cuffmerge}}

\newglossaryentry{cummeRbund}{name={cummeRbund},
description={is an R package that is designed to aid and simplify the task of analyzing Cufflinks RNA-Seq output (\url{http://compbio.mit.edu/cummeRbund/})}}

\newglossaryentry{blacktie}{name={Blacktie},
description={a python application written to facilitate the automation of the \gls{RNA-Seq} using the \gls{TuxProt}.}}

\newglossaryentry{SunGridEngine}{name={Sun Grid Engine},
description={a popular job scheduling system used to coordinate the efficient use of jobs submitted to a cluster computing environment}}


\newglossaryentry{Pcc}{name={Pearson correlation coefficient},
description={a measure of the linear correlation between two variables, resulting in a value between +1 and −1, where 1 is perfect positive correlation, 0 is no correlation, and −1 is perfect negative correlation}}


\newglossaryentry{functional-genomics}{name={functional genomics},
description={an approach to molecular biology that attempts to incorporate much of the varied wealth of data produced by modern ``Omics'' projects to describe the functions and interactions of the entire genome; it specifically focuses on dynamic aspects of the genome such as transcription, translation, regulatory, and protein–protein (or protein-DNA) interactions, as opposed to the static aspects of the genomic information such as DNA sequence or structures}}

\newglossaryentry{Git}{name={Git},
description={In software development, Git is a distributed revision control and source code management system with an emphasis on speed.
Every Git working directory is a full-fledged repository with complete history and full version tracking capabilities, not dependent on network access or a central server.
Git is free software distributed under the terms of the GNU General Public License version 2. \url{http://en.wikipedia.org/wiki/Git_(software)}}}

\newglossaryentry{TuxProt}{name={Tuxedo Protocol},
description={an integrated RNA-Seq analysis protocol consisting of the following co-developed programs: Bowtie, Tophat, Cufflinks, Cuffmerge, Cuffdiff, and CummeRbund.  An overview figure and further description can be found here: \url{http://www.nature.com/nprot/journal/v7/n3/fig_tab/nprot.2012.016_F2.html}}}

\newglossaryentry{monophyletic}{name={monophyletic},
description={having developed from a single ancestral source}}

\newglossaryentry{anthropophilic}{name={anthropophilic},
description={the characteristic of preferring to feed on or live amidst humans}}

\newglossaryentry{anautogeny}{name={anautogeny},
description={the characteristic of requiring a bloodmeal to successfully mature a clutch of eggs}}


\newglossaryentry{hematophagy}{name={hematophagy},
description={the characteristic of feeding on blood}}

\newglossaryentry{transgene}{name={transgene},
description={a gene that has been transfered from one organism to another}}

\newglossaryentry{germline}{name={germline},
description={the cells responsible for eventually forming the sperm or egg cells in a mature sexual organism}}


\newglossaryentry{phic31}{name={\phic},sort={phic31},
description={a site-specific bacteriophage recombinase that performs precise, unidirectional recombination between two sites called attB and attP}}

\newglossaryentry{population-reduction}{name={population reduction},
description={a vector control strategy that focuses on reducing the local population of a disease vector with the goal of reducing the probability that an infectious individual will have an opportunity to participate in a transmission event}}

\newglossaryentry{population-modification}{name={population modification},
description={a novel vector control strategy that aims to drive a transmission-refractory trait into a local vector population with the goal of preventing individuals from every becoming infectious}}

\newglossaryentry{hemolymph}{name=hemolymph,
description={A fluid analogous to blood that constitutes the circulatory system of most invertebrates}}

\newglossaryentry{gene-drive}{name={gene-drive},
description={molecular genetic tactics that cause a linked trait or gene to spread through a population at faster rates than expected based on fitness alone; generally operating independently of natural selection and genetic drift}}
	
\newglossaryentry{transposon}{name=transposon,see=transposable-element}
\newglossaryentry{transposable-element}{name=transposable element,
description={relatively short regions of DNA in an organism's genome that have the ability to cut (or copy) themselves from the genome and insert themselves into a new location in the genome}}
	
\newglossaryentry{effector-gene}{name={effector gene},
description={a gene that will cause the desired change in the environment of the mosquito or other vector.  An example might be a gene that codes for a protein that targets and destroys the pathogen}}

\newglossaryentry{prevalence}{name=prevalence,
description={the number or proportion of cases or events or attributes among a given population}}

\newglossaryentry{incidence}{name=incidence,
description={a measure of the frequency with which new cases of illness, injury, or other health condition occurs among a population during a specified period}}

\newglossaryentry{midgut}{name=midgut,
description={the location in the mosquito's digestive system where the bloodmeal is processed}}

\newglossaryentry{vector}{name=vector,
description={an organism that facilitates the transfer of pathogens from one host to another}}
