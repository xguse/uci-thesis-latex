\chapter{Introduction}

\section{Purpose}

Introduce core concepts relating to my project such as:

\begin{enumerate}
\item Importance of Dengue and Malaria
\item Transmission cycle dependent on Mosquitoes
\item Limited effectiveness (medical/finacial) of treatment of the diseases in rural poor human communities
\item Vector Control Strategies
\item How Transgenesis modifies the vector control landscape
\item Elements of Successful Transgenesic Vector Control
\item Introduction to control of transcription and imoprtance of \glspl{CRE}/\glspl{CRM}
\end{enumerate}


\section{Dengue fever and malaria are mosquito borne diseases}

\synopsis{
\begin{itemize}
\item there are many but we focus on Dengue Fever and Malaria
\item causative agents are RNA viruses and Plasmodium protists, respectively
\item people don't give people these two diseases
\item mosquitoes generally don't give mosquitoes these diseases
\end{itemize}
}

There are many mosquito borne diseases that afflict the world's population; \alert{however, the James Lab focuses on two of the major ones: dengue fever and malaria}.
The causative agents of these two diseases are a suite of four species of RNA viruses (Dengue Virus 1-4) and a suite of five species of protist from the genus \textit{Plasmodium}, respectively.
However, in any \textbf{single} geographic location they are generally transmitted by a single species of mosquito \alert{CITEME}. 

\alert{TABLE HERE}

The fact that both of these diseases are caused by multiple pathogen species complicates efforts to prevent human disease: particularly when addressed from the perspective of providing therapeutic or prophylactic care directly to human patients.

\section{The transmission cycles of dengue fever and malaria require mosquitoes}

Another very important aspect of both of these diseases is that people do not become infected through interaction with other people, and mosquitoes do not become infected through interaction with other mosquitoes.
The transmission cycles transmission of both diseases require that the pathogen be passed from human to mosquito or from mosquito to human.
This pattern has important consequences that we can exploit to prevent the transmission cycle from producing as many infected individuals.

\subsection{Transmission cycle}

\begin{enumerate}
\item a female mosquito bites infected person
\item the pathogen is taken into mosquito midgut with bloodmeal
\item the pathogen must escape the midgut and gain access to the mosquito's circulatory system (the animal is now \textbf{infected})
\item the pathogen must gain access to the mosquito's salivary glands (the animal is now \textbf{infectious})
\item the infectious female mosquito bites an uninfected human and pathogens in her saliva are introduced to the person's body
\end{enumerate}
While the specifics of how viruses and protists live and reproduce in mosquitoes and humans are quite different, the transmission cycles of dengue and malaria are very similar in their major events.
The transmission of either pathogen from one infected human to another can both be summarized into five fundamental events.
First, a female mosquito feeds on an infected human's blood taking up the pathogen as well.
The bloodmeal is digested in the mosquito's midgut, \alert{and as you might expect, the midgut is the first stop for the pathogen too.}
Escape of the midgut is the first critical step for the survival of the pathogen inside the mosquito.
If it is trapped in the midgut, it will eventually be passed as waste after the bloodmeal is digested.
If it manages to escape to the \gls{hemolymph}, the pathogen must gain access to the mosquito's salivary glands if it is to infect another human.
If this step is unsuccessful, then the mosquito, while itself \textbf{infected}, is not \textbf{infectious} to other humans.
For this, the pathogen must be injected into the bloodstream of the next human along with the contents of the mosquito's saliva.

\section{Public health: \emph{multiple targets}}

\synopsis{
\begin{itemize}
\item The specifics of the transmission cycle of these diseases provides multiple targets for public health interventions
\item focus on the human
\item focus on the vector
\end{itemize}
}

Because the arrow of transmission always\footnote{While some mosquito
  to non-human vertebrate transmission may occur, it is not thought to
  be sufficient to maintain the pathogen if the mosquito:human loop is
  broken.} points from mosquito to human or from human to mosquito, if either of those arrows are broken, the cycle will collapse and the area would eventually be cleared of the pathogen provided the intervention is maintained.
This model suggests that there are two primary targets for public health interventions aiming to reduce the \gls{incidence}
(followed by \gls{prevalence}) of infection in a population.

\begin{enumerate}
\item
  prevent mosquitoes from infecting humans
\item
  prevent humans from infecting mosquitoes
\end{enumerate}

However, many of the interventions can not be clearly divided into addressing purely the human or vector side of the cycle.
For example, any efforts to reduce the number of bites that humans receive from mosquitoes affects both the probabilities of the vector \emph{and} the human becoming infected.
This is one reason that vector control rather than purely human-based interventions are almost always part of prevention strategies.
For the purposes of this document, I will define human-based interventions to be those that are directly administered to the human's body.
Essentially: medical interventions.

It should be noted that in this respect, the long term goals of public health are more focused on prevention than the treatment of acute cases.

Of course sick people need to be treated, and finding and clearing people who are infected is a part of preventing mosquitoes from acquiring the pathogen from people.
However, in the long game it is much more effective to prevent the infection.
These efforts are what we will focus on.

\subsection{Options for the humans}

Because of the definition of human-based interventions that I am using, there are relatively few effective options in this category for malaria and dengue fever.
Normally, this section would include vaccines and swift, effective patient identification and treatment to clear the infection.
\alert{For reasons that will be discussed in a later post, these interventions simply do not exist on the market or in a cost-effective form applicable to the isolated and impoverished areas that are most affected.}

\subsection{Conventional options for the vectors}

Because medical options are generally quite scarce, most attention in the field is directed toward controlling access of the vector populations to human contact.
This can include removal of nearby mosquito breeding sites (usually standing water), spraying of insecticides, introduction of biological predators, and/or bed nets, etc.
One fairly novel approach that has implications for the next post in this series is sterile insect technique.



\subsection{Sterile Insect Technique}
\href{http://en.wikipedia.org/wiki/Sterile\_insect\_technique}{Sterile insect technique (SIT)} exploits a peculiar aspect of some insects' reproductive behavior. 
In many insects, it is only (or at least
primarily) the first mating event that ``matters''.
Subsequent mating events contribute little to no genetic material to the females progeny, \textbf{even when the first event involves a sterile male}.
This means that if massive numbers of sterilized males are introduced into a native population, any wild female that mates first with one of the sterile males will be effectively sterilized herself.
This can have dramatic effects on the local population.
For a famously effective SIT campaign, look up \href{http://goo.gl/DF7bv}{screwworm eradication} on google.


\section{Transgenic options for the vectors}

Most conventional vector control strategies involve what might be termed \textbf{vector population reduction}.
As we get into what transgenics can do for vector control, we will see that in addition to population reduction we have a new strategy available which has quite exciting implications to the sustainability of the vector control aspect of public health interventions for dengue and malaria.
This could be called \textbf{vector population conversion}.

\section{Introducing Inheritable Transgenes into the Mosquito Genome}

Before one can employ transgenic methods for vector control, one must of course have the ability to make transgenic vectors.
For mosquitoes, this is quite a bit more labor intensive than for other fly species like \emph{Drosophila melanogaster}.
However, it has been accomplished, and protocols to reliably introduce transgenes into many mosquito species now exist.
Unfortunately, the relative difficulty is still high, and it takes roughly 2-3 months before a stable line is established.
So while this step can be filed under ``accomplished'', work is constantly being done to improve its ease and efficiency.

The method of gene insertion into the genome generally relies on re-purposing the activities of small, ``parasitic'', mobile DNA elements called \glspl{transposon} or \glspl{transposable-element}.
These elements have the ability to cut themselves out of one part of the DNA and re-insert themselves into another.
We can engineer them in the lab to carry our genes instead of the genes that they usually contain.
This allows our genes to be inserted into the genome instead of the contents of the original transposon.

Until recently, most if not all transposons used inserted the genes into more-or-less random regions, and in single, double or \textbf{multiple} copies and/or locations \cite{Adelman2004}, \cite{Sethuraman2007}.
This makes characterizing the effects of the effector genes complicated.
However, with the successful implementation of the \href{http://en.wikipedia.org/wiki/Site-specific\_recombinase\_technology\#PhiC31\_Integrase}{phic31 site specific integration system} in \emph{Aedes aegypti} and \emph{Anopheles stephensi} \cite{Thorpe1998}, \cite{Nimmo2006}, \cite{Isaacs2012}, we can now reliably target specific, known regions of the genome for transgene integration of a single copy.

The ``heritable'' part is achieved through which cells take up the transgene.
Only certain types of cells have the ability to project their genetic code into the next generation.
In order to have a transgene be heritable by the following generations we must make sure that at least egg cells).

A peculiarity about fly development is that as the embryo is developing the cells that will become the germ line always position themselves at one end (or pole) of the embryo.
They earned the moniker ``pole cells''.
This makes targeting these cells for the transgene feasible.
When the solution containing the transgene construct is injected into the embryos, it is done where the pole cells are.
This means that the first set of mosquitoes that might be fully transgenic is not the set that you injected but the first generation of offspring that your injected mosquitoes produce.

\section{Effector Genes}

With the ability to integrate transgene constructs into the vectors' genomes, we need constructs that carry genes that will effect the response that we desire in the vector.
Two broad strategies are commonly applied when designing an anti-transmission phenotype.

\subsection{Vector Population Reduction:}

The goal of vector population reduction is the same as the conventional vector control modalities.
\emph{Reduce the size of the vector population to decrease the probability of infectious interactions between vectors and humans.}

\textbf{Conventional tactics} include removal of breeding sites through draining of swamps, removal of items that store water (tires, buckets, etc) where larvae develop from populated areas, and of course, chemical insecticidal campaigns.
Another, more recent tactic is sterile insect technique.

A \textbf{transgenic tactic} for the population reduction strategy is illustrated by \href{http://www.oxitec.com/}{Oxitec's} \emph{Aedes aegypti} strain \href{http://www.oxitec.com/health/our-products/aedes-agypti-ox3604c/}{OX3604C} developed with support from the \href{http://www.grandchallenges.org/Pages/Default.aspx}{Bill and Melinda Gates Foundation's Grand Challenges for Global Health Initiative}.
OX3604C, represents a female-specific \gls{RIDL} approach which uses a poison transgene that kills any female adult expressing the transgene unless the female is being fed the ``antidote'' through its water supply.

Releasing enough of these mosquitoes will affect the local mosquito species population in a way that is analogous to \gls{SIT}.

\subsection{Vector Population Replacement (Conversion):}

Vector population conversion is a novel strategy for vector control that only exists within the context of genetically altering the vectors.
This strategy actually promises to be the most long-lasting vector-focused intervention; because unlike \textbf{all} vector population reduction tactics, there is a theoretical point in a conversion intervention (the anti-transmission trait works and is present in local populations at near 100\%) when the human interaction can be ceased but the intervention continues to function.
For population reduction to approach this result, the vector species must not simply be eliminated from the local area, but approach elimination on a continental scale, or more realistically achieve global eradication.

The reason is that these mosquitoes (especially \emph{Aedes aegypti}) can and do travel long distances, as eggs or larvae, in the backs of trucks (between villages) or in the pools of water collected in super-tankers (transcontinentally).
So a local village is only free of the vectors until more migrate into the area.
But in a conversion scenario, those migrants mate with the local vector population, and their offspring are assimilated into transmission-deficient mosquitoes.
The protection of the local village can be preserved \emph{even if some surrounding villages fail to maintain control of their mosquitoes}.

An example of a \textbf{transgenic tactic} for population conversion of \emph{Aedes aegypti} into a transmission-deficient phenotype involves an effector gene that codes for double stranded portion of the target dengue virus as RNA \cite{Franz2006}, \cite{Mathur2010}.
Because most animal cells have a system that detects and degrades double stranded RNA
\footnote{Double stranded RNA generally signals that a virus is active in the cell.}
in a sequence specific manner, this primes the mosquito cells' antiviral response to specifically attack the dengue virus if the effector gene is expressed in the cell before it gets infected.

\section{Controlling When and Where the Effector Genes are Expressed}

Even armed with an effector gene that clears 100\% of the pathogen 100\% of the time, you will not be successful in limiting transmission if it is not turned on in the right time and place.
If your effector gene works best when the pathogen is in the midgut, but your gene is only expressed in the antennae, you have wasted your time.

The region of DNA directly before the sequence of the gene is \emph{usually} the most influential determinant of the pattern of expression and is referred to as the promoter.
It determines when the gene will be turned on (let's say directly following the ingestion of a bloodmeal), and in which tissue type (let's say the midgut).
The way that this is accomplished is due to the binding of special proteins called transcription factors that recognize specific DNA sequences.
Once bound to the promoter they recruit the special machinery needed for the gene to be turned on.

If we wanted to control a transgene in a specific way (turned on after a bloodmeal in the midgut), one way to go about it would be to identify genes in the mosquito that already have a similar expression pattern to the ideal that you want.
Then we could copy the promoter from that gene and paste it in front of our transgene
\footnote{The story is \textbf{of course} more complicated than this, and the replication of the
    original expression pattern may not always be perfect with this simplified method; however,
    it is suitably explanatory for our purposes at the moment.}.
Because we will have replicated the specific \glspl{TFBS} that control the original gene, our transgene should inherit a very similar expression pattern.

This is a very common process used to engineer the expression patterns of real transgenes in mosquitoes.
In \cite{Moreira2000}, the promoter sequences of a gene called carboxypeptidase (normally expressed in the midgut after a meal to help digest it) from \emph{Aedes aegypti} and \emph{Anopheles gambiae} were pasted in front of a transgene that causes the cells that express the gene to light up.
This type of transgene is called a \gls{reporter gene} because it allows the researchers to visualize the activity of the promoter used to drive its expression.

From the abstract of \cite{Moreira2000}:

\begin{quote}
Six independent transgenic lines were obtained with the AeCP construct and one with the
AgCP
construct.
Luciferase mRNA and protein were abundantly expressed in the
guts of transgenic mosquitoes in four of the six AeCP lines and in the
AgCP line.
Expression of the reporter gene was gut-specific and reached
peak levels at about 24 h post-blood ingestion.
\end{quote}

\section{Achieving Swift, Dramatic, and Clinically Meaningful Effects on the Native Vector Population}

In many ways, this is the most difficult part of the puzzle.
In order for the transgene to have its self-sustaining properties as well as achieve effective anti-transmission results, it must spread through the native mosquito population to the point that the percent of individuals possessing the gene approaches 100\%.

Optimistically assuming that the transgene carries a negligible fitness cost
\footnote{By fitness cost/advantage here, I mean that the transgene
    causes the mosquitoes that inherit it to be either less or more
    successful at producing offspring, respectively.},
or even a slight fitness advantage, achieving near 100\% conversion could take decades.
Funding terms for efforts like this in poor nations can be closer to 5 years or less.
To enable the population conversion strategies to work, we must come up with genetic ``tricks'' that cause the gene to spread through a mosquito population \textbf{much} faster than could happen naturally.
Efforts to discover or design this \gls{gene-drive} system are an on going and active area in this field.