

\chapter{Results} \label{chap:4}

\section{Preliminary data preparation}

\section{gFunc-based analysis}

\begin{figure}[hp]
%
\subcaptionbox{\label{fig:ecr-pair-ptci-hists-base}}
{\includegraphics[width=.5\linewidth]{figures/figs/ecr_team_ptci_20130918_orthodb7/pairwise_ptci_hist.pdf}}
% 
\subcaptionbox{\label{fig:ecr-pair-ptci-hists-rcum-hist}}
{\includegraphics[width=.5\linewidth]{figures/figs/ecr_team_ptci_20130918_orthodb7/pairwise_ptci_cum_hist.pdf}}
% 
\subcaptionbox{\label{fig:ecr-pair-ptci-hists-fdr}}
{\includegraphics[width=.5\linewidth]{figures/figs/ecr_team_ptci_20130918_orthodb7/pairwise_ptci_fdr.pdf}}
% 
% 
\caption[Pairwise 20E-PTCI results]{\sf \textbf{Pairwise PTCI results for the \gls{20E}-\gls{TFBS} group}:\\
\textbf{(A)} Histogram of mean PTCI results vs null distributions.
\textbf{(B)} Reverse Cumulative histogram of mean PTCI results vs null distributions.
\textbf{(C)} False discovery rate vs PTCI threshold.}
\label{fig:ecr-pair-ptci-hists}
\end{figure}
% EcR-pair (Figure \ref{fig:ecr-pair-ptci-hists})

\begin{figure}[hp]
%
\subcaptionbox{\label{fig:ecr-mean-ptci-hists-base}}
{\includegraphics[width=.5\linewidth]{figures/figs/ecr_team_ptci_20130918_orthodb7/mean_ptci_hist.pdf}}
% 
\subcaptionbox{\label{fig:ecr-mean-ptci-hists-rcum-hist}}
{\includegraphics[width=.5\linewidth]{figures/figs/ecr_team_ptci_20130918_orthodb7/mean_ptci_cum_hist.pdf}}
% 
\subcaptionbox{\label{fig:ecr-mean-ptci-hists-fdr}}
{\includegraphics[width=.5\linewidth]{figures/figs/ecr_team_ptci_20130918_orthodb7/mean_ptci_fdr.pdf}}
% 
% 
\caption[Mean 20E-PTCI results]{\sf \textbf{Mean PTCI results for the \gls{20E}-\gls{TFBS} group}:\\
\textbf{(A)} Histogram of mean PTCI results vs null distributions.
\textbf{(B)} Reverse Cumulative histogram of mean PTCI results vs null distributions.
\textbf{(C)} False discover rate vs PTCI threshold.}
\label{fig:ecr-mean-ptci-hists}
\end{figure}
% EcR-mean (Figure \ref{fig:ecr-mean-ptci-hists})

\begin{figure}[hp]
%
\subcaptionbox{\label{fig:insect-pair-ptci-hists-base}}
{\includegraphics[width=.5\linewidth]{figures/figs/jaspar_insect_ptci_20130918_orthodb7/pairwise_ptci_hist.pdf}}
% 
\subcaptionbox{\label{fig:insect-pair-ptci-hists-rcum-hist}}
{\includegraphics[width=.5\linewidth]{figures/figs/jaspar_insect_ptci_20130918_orthodb7/pairwise_ptci_cum_hist.pdf}}
% 
\subcaptionbox{\label{fig:insect-pair-ptci-hists-fdr}}
{\includegraphics[width=.5\linewidth]{figures/figs/jaspar_insect_ptci_20130918_orthodb7/pairwise_ptci_fdr.pdf}}
% 
% 
\caption[Pairwise insect-PTCI results]{\sf \textbf{Pairwise PTCI results for the insect-\gls{TFBS} group}:\\
\textbf{(A)} Histogram of mean PTCI results vs null distributions.
\textbf{(B)} Reverse Cumulative histogram of mean PTCI results vs null distributions.
\textbf{(C)} False discover rate vs PTCI threshold.}
\label{fig:insect-pair-ptci-hists}
\end{figure}
% Insect-pair (Figure \ref{fig:insect-pair-ptci-hists})

\begin{figure}[hp]
% /home/gus/Dropbox/repos/git/uci-thesis-latex/figures/figs/jaspar_insect_ptci_20130918_orthodb7/mean_ptci_cum_hist.pdf
\subcaptionbox{\label{fig:insect-mean-ptci-hists-base}}
{\includegraphics[width=.5\linewidth]{figures/figs/jaspar_insect_ptci_20130918_orthodb7/mean_ptci_hist.pdf}}
% 
\subcaptionbox{\label{fig:insect-mean-ptci-hists-rcum-hist}}
{\includegraphics[width=.5\linewidth]{figures/figs/jaspar_insect_ptci_20130918_orthodb7/mean_ptci_cum_hist.pdf}}
% 
\subcaptionbox{\label{fig:insect-mean-ptci-hists-fdr}}
{\includegraphics[width=.5\linewidth]{figures/figs/jaspar_insect_ptci_20130918_orthodb7/mean_ptci_fdr.pdf}}
% 
% 
\caption[Mean insect-PTCI results]{\sf \textbf{Mean PTCI results for the insect-\gls{TFBS} group}:\\
\textbf{(A)} Histogram of mean PTCI results vs null distributions.
\textbf{(B)} Reverse Cumulative histogram of mean PTCI results vs null distributions.
\textbf{(C)} False discovery rate vs PTCI threshold.}
\label{fig:insect-mean-ptci-hists}
\end{figure}
% Insect-mean (Figure \ref{fig:insect-mean-ptci-hists})

These results pertain to the second stage of the approach described in Chapter \ref{chap:3} (Figure \ref{fig:approach-chart}: \textit{yellow box}).

\paragraph*{Complementary sets of TFBS models:}
\gls{PTCI} data was calculated using two complementary sets of \gls{TFBS} models.
%
One set focuses the \gls{PTCI} on genes that have been associated with \gls{20E} and its nuclear receptor (\PTCIe) (Figures \ref{fig:ecr-pair-ptci-hists} and \ref{fig:ecr-mean-ptci-hists}), and a second provides a general focus on \gls{TFBS} models provided by JASPAR that are defined in insects at large (\PTCIi) (Figures \ref{fig:insect-pair-ptci-hists} and \ref{fig:insect-mean-ptci-hists}).
%
\FDR\ estimation was used to determine a suitable \PTCI\ value to use as a threshold for further investigating particular 3-way 1:1 ortholog sets.
%
It was determined that in both \gls{TFBS} model data-sets a mean \PTCI\ threshold of 0.95 yielded an \FDR\ of approximately 15\% and maximized the number of genes for further classification (Figures \ref{fig:ecr-mean-ptci-hists-fdr} and \ref{fig:insect-mean-ptci-hists-fdr}).
%
This threshold produced 666 and 708 genes in the \PTCIi\ and \PTCIe\ sets, respectively.
%
The union of these gene-sets was 930 genes or 310 genes from each species.
%
The intersection was 444 or an overlap of 66\% of the \PTCIi\ genes and 62.7\% of the \PTCIe\ genes.
%
The unique proportion contributed by each set are 33\% and 37.3\%, respectively.

\paragraph*{Mean vs pairwise PTCI:}
Comparing the \FDR\ values of the pairwise- and mean-\PTCI\ scores at the 0.95 threshold reveals the impact of considering information from all 3-way 1:1 orthologs simultaneously rather than species-pair by species-pair (Figures \ref{fig:ecr-pair-ptci-hists-fdr} vs \ref{fig:ecr-mean-ptci-hists-fdr} and Figures \ref{fig:insect-pair-ptci-hists-fdr} vs \ref{fig:insect-mean-ptci-hists-fdr}).
%
The quality, as judged by \FDR, is strikingly improved by including the relationships of all three species weighted by their pairwise evolutionary distance: an improvement of approximately 30 percentage points in both cases.

\section{Characterization of Results}
These results pertain to the third stage of the approach described in Chapter \ref{chap:3} (Figure \ref{fig:approach-chart}: \textit{blue box}).

\subsection{Functional annotations (before k-means clustering)}

Functional annotations were obtained for the 930 genes using \gls{Argot2} (Section \ref{chap:3-sec:characterization-of-results}).
%
%% How many of the 930 were assigned at least one annotation >= 200? --> 760 (Aa:253 ,Ag:254 , Cq:253 )
%% How many genes got better than 2000? --> 452 (Aa: 154, Ag:153 , Cq:145 )
Of these, 760 were assigned, by \gls{Argot2}, at least one annotation with a \gls{TS} greater than or equal to 200\footnote{This is the threshold that is suggested by the developers based on their in-house benchmarking \url{http://www.medcomp.medicina.unipd.it/Argot2/help/argot\_scores.php\#ts}} (\Aa: 253, \Ag: 254, \Cq: 253).
%
452 were assigned annotations with \glspl{TS} greater than or equal to 2000 (\Aa: 154, \Ag: 153, \Cq: 145).
%
This indicates that on the whole, most of the 930 genes produced by the \gls{gFunc} process were assigned annotations of a quality at least as stringent as used by the developers of \gls{Argot2}.

%%%%%%%%%%%%%%%%%%%%%


\begin{landscape}

    \begin{figure}[h]
    \centering
    \includegraphics[width=\linewidth]{figures/figs/ecr_and_insects_ptci_20130918_orthodb7/23clusters_ptci_0_95_orthodb7.pdf}
    \caption[\Ag\ clustered abundance profiles]{\sf \textbf{\Ag\ clustered abundance profiles.} \\ 
    Clusters were generated using k-means clustering as implemented in Biopython version 1.62 \cite{Cock2009}.  Abundance profile data was log transformed after one FPKM was added to all data to remove zeros ($log_{10}(\mathrm{FPKM}+1)$).  K-means was then applied using the arithmetic mean as the center for cluster definition.  The data displayed here is the gene-wise standardization of the \textbf{raw} FPKM data such that each profile has mean = 0 and standard deviation = 1. Each median abundance profile is marked by a thick gray line. \textbf{Panel Titles} - ID: cluster identifier | members: number of genes in cluster | percentage of genes represented in all clusters. \textbf{Time Points} - \gls{NBF}, 4, 6, 8, 10 h \gls{PBM}.
}
    \label{fig:23-clusters}
    \end{figure}
    
\end{landscape}


\begin{figure}[hp]
% 
\subcaptionbox{\label{fig:cluster4-Aa}}
{\includegraphics[width=.5\linewidth]{figures/figs/ecr_and_insects_ptci_20130918_orthodb7/upAfter4_gene_profiles_from_cummerbund/Aa_upAfter4_cls4_Ag_target_FPKMs_vb_orthos.pdf}}
%
\subcaptionbox{\label{fig:cluster4-Ag}}
{\includegraphics[width=.5\linewidth]{figures/figs/ecr_and_insects_ptci_20130918_orthodb7/upAfter4_gene_profiles_from_cummerbund/Ag_upAfter4_cls4_Ag_target_FPKMs_vb_orthos.pdf}}
%
\subcaptionbox{\label{fig:cluster4-Cq}}
{\includegraphics[width=.5\linewidth]{figures/figs/ecr_and_insects_ptci_20130918_orthodb7/upAfter4_gene_profiles_from_cummerbund/Cq_upAfter4_cls4_Ag_target_FPKMs_vb_orthos.pdf}}
% 
\caption[Orthologs of cluster 4]{\sf \textbf{Orthologs of cluster 4 (up after 4h):}\\
The same color scheme is used for each species which means that orthologs are given the same color in all three panels.
The thick, transparent gray line represents the median \gls{mAP} for the panel.
\textbf{(A)} \Aa.
\textbf{(B)} \Ag.
\textbf{(C)} \Cq.
}\label{fig:cluster4}
\end{figure}

\begin{figure}[hp]
% 
\begin{subfigure}[t]{.5\linewidth}
\includegraphics[width=\linewidth]{figures/figs/ecr_and_insects_ptci_20130903/upAt4_gene_profiles_from_cummerbund/Aa_upAt4_cls6_Ag_target_FPKMs_vb_orthos.pdf}
\caption{}
\label{fig:cluster6-Aa}
\end{subfigure}%
%
\begin{subfigure}[t]{.5\linewidth}
\includegraphics[width=\linewidth]{figures/figs/ecr_and_insects_ptci_20130903/upAt4_gene_profiles_from_cummerbund/Ag_upAt4_cls6_Ag_target_FPKMs_vb_orthos.pdf}
\caption{}
\label{fig:cluster6-Ag}
\end{subfigure}
% 
\begin{subfigure}[t]{.5\linewidth}
\includegraphics[width=\linewidth]{figures/figs/ecr_and_insects_ptci_20130903/upAt4_gene_profiles_from_cummerbund/Cq_upAt4_cls6_Ag_target_FPKMs_vb_orthos.pdf}
\caption{}
\label{fig:cluster6-Cq}
\end{subfigure}
% 
\caption[Orthologs of cluster 6]{\sf \textbf{Orthologs of cluster 6:}\\

\textbf{(A)} \Aa.
\textbf{(B)} \Ag.
\textbf{(C)} \Cq.
\todo[inline,caption={Finish Fig \ref{fig:cluster6}}]{ 
\begin{itemize}
    \item the text
    \item double gene names
    \item fix fig widths
\end{itemize}}
}
\label{fig:cluster6}
\end{figure}
\input{figures/cluster16_AaAgCq}
\input{figures/cluster22_AaAgCq}

\subsection{K-means clustering}

K-means clustering ($k=23$) was applied to the 310 genes from \Ag\ to attempt to partition the 3-way 1:1 ortholog sets based on \gls{mAP} similarity.
%
\Ag\ was selected because it shares approximately equidistant evolutionary divergence to both other species.
%
Four of the resulting clusters were chosen for further characterization (Figure \ref{fig:23-clusters}: \textit{Cluster IDs 4, 6, 16, and 22}) because they have expression patterns that may be regulated by a noted pulse of \gls{20E} that occurs around 4h \gls{PBM}\footnote{See Section \ref{chap:3-sec:clustering-of-filtered-ortholog-sets} for more information.}.
%




% cls4
\input{tables/cls4_process.tex}
Table \ref{tab:cls4-process}
% Booktabs require to add \usepackage{booktabs} to your document preamble
\begin{table}[hp]
\begin{center} \sf
\begin{tabular}{p{.7\textwidth}r}
\toprule
\textbf{Name}                                                                                  & \textbf{Total Score (mean)} \\ \midrule
translation initiation factor activity                                                         & 18796.17                    \\ % translation/ TOR?
protein transporter activity                                                                   & 13663.10                    \\
signal recognition particle binding                                                            & 9157.88                     \\
endoplasmic reticulum signal peptide binding                                                   & 7943.20                     \\ % secretion 
Rab GDP-dissociation inhibitor activity                                                        & 7361.38                     \\
hydrolase activity, acting on acid anhydrides, catalyzing transmembrane movement of substances & 6647.91                     \\
hydrogen ion transmembrane transporter activity                                                & 6278.87                     \\
ferrochelatase activity                                                                        & 4990.12                     \\
GTP binding                                                                                    & 4006.06                     \\
%enzyme binding                                                                                 & 3697.54                     \\
saccharopine dehydrogenase (NAD+, L-glutamate-forming) activity                                & 3453.61                     \\
zinc ion binding                                                                               & 3215.15                     \\
glutamate-ammonia ligase activity                                                              & 2901.02                     \\
RNA binding                                                                                    & 2799.62                     \\
ATPase activity, coupled to transmembrane movement of substances                               & 2678.79                     \\
pyridoxal phosphate binding                                                                    & 2586.25                     \\
transporter activity                                                                           & 2044.78                     \\
saccharopine dehydrogenase (NADP+, L-lysine-forming) activity                                  & 1890.88                     \\
threonine-type endopeptidase activity                                                          & 1842.63                     \\
GTPase activator activity                                                                      & 1780.38                     \\
transaminase activity                                                                          & 1686.97                     \\
aminopeptidase activity                                                                        & 1592.00                     \\
endopeptidase activity                                                                         & 1546.38                     \\
nucleoside-triphosphatase activity                                                             & 1372.39                     \\
ATPase activity                                                                                & 1369.37                     \\
hydrolase activity                                                                             & 1352.33                     \\
lyase activity                                                                                 & 1299.81                     \\
ubiquitin-protein ligase activity                                                              & 1261.81                     \\
oxidoreductase activity                                                                        & 1237.14                     \\
nucleotide binding                                                                             & 1166.69                     \\
GTPase activity                                                                                & 1110.62                     \\
protein binding                                                                                & 1013.21                     \\
oligopeptide transporter activity                                                              & 927.01                      \\
symporter activity                                                                             & 915.24                      \\
ligase activity                                                                                & 870.70                      \\
metallopeptidase activity                                                                      & 809.99                      \\
metal ion binding                                                                              & 794.06                      \\
peptidase activity                                                                             & 792.59                      \\
proton-transporting ATPase activity, rotational mechanism                                      & 702.75                      \\
% ATP binding                                                                                    & 662.73                      \\
% catalytic activity                                                                             & 438.09                      \\
transferase activity                                                                           & 319.82                      \\
dipeptide transporter activity                                                                 & 217.58                      \\ \bottomrule
\end{tabular}
\end{center}

\caption[Cluster 4 top function GO terms]{\sf \textbf{Cluster 4 top function GO terms}}
\label{tab:cls4-function}
\end{table}
Table \ref{tab:cls4-function}
% \input{tables/cls4_cellular.tex}
% Table \ref{tab:cls4-cellular}

% cls6
\input{tables/cls6_process.tex}
Table \ref{tab:cls6-process}
\input{tables/cls6_function.tex}
Table \ref{tab:cls6-function}
% % Booktabs require to add \usepackage{booktabs} to your document preamble
\begin{table}[h]
\begin{center} \sf
\begin{tabular}{@{}lr@{}}
\toprule
\textbf{Name}                                      & \textbf{Total Score (mean)} \\ \midrule
cytosol                                            & 11598.25                    \\
mitochondrial matrix                               & 5271.13                     \\
nucleus                                            & 3895.71                     \\
plastid                                            & 2414.67                     \\
chloroplast stroma                                 & 2058.69                     \\
chloroplast                                        & 1874.07                     \\
nuclear pore                                       & 1453.16                     \\
positive transcription elongation factor complex b & 1281.93                     \\
mitochondrion                                      & 1208.52                     \\
cytoplasm                                          & 1095.77                     \\
intracellular                                      & 970.52                      \\
signal recognition particle                        & 809.92                      \\
focal adhesion                                     & 735.87                      \\
ribonucleoprotein complex                          & 669.27                      \\
cell junction                                      & 666.19                      \\
spliceosomal complex                               & 587.46                      \\
synapse                                            & 481.27                      \\
cell projection                                    & 442.30                      \\
dynein complex                                     & 343.74                      \\
plasma membrane                                    & 317.82                      \\
lamellipodium membrane                             & 302.49                      \\
filamentous actin                                  & 260.41                      \\
nuclear body                                       & 250.79                      \\
lamellipodium                                      & 227.79                      \\
actin filament                                     & 226.01                      \\
membrane                                           & 216.49                      \\
mitochondrial inner membrane                       & 216.01                      \\ \bottomrule
\end{tabular}
\end{center}

\caption[Cluster 6 top cellular GO terms]{\sf \textbf{Cluster 6 top cellular GO terms}}
\label{tab:cls6-cellular}
\end{table}
% Table \ref{tab:cls6-cellular}

% cls16
% Booktabs require to add \usepackage{booktabs} to your document preamble
\begin{table}[hp]
\begin{center} \sf
\begin{tabular}{p{.7\textwidth}r}
\toprule
\textbf{Name}                                 & \textbf{Total Score (mean)} \\ \midrule
one-carbon metabolic process                  & 8498.68                     \\
carbohydrate metabolic process                & 4077.43                     \\
translation                                   & 3505.77                     \\
regulation of Rho protein signal transduction & 1336.90                     \\
homophilic cell adhesion                      & 1316.75                     \\
cell adhesion                                 & 851.76                      \\
autophagy                                     & 686.01                      \\
apoptotic process                             & 670.22                      \\
regulation of myelination                     & 618.20                      \\
protein phosphorylation                       & 542.93                      \\
phosphorylation                               & 395.02                      \\
metabolic process                             & 286.58                      \\
protein transport                             & 262.15                      \\
protein ubiquitination                        & 242.49                      \\
intracellular signal transduction             & 212.22                      \\ \bottomrule
\end{tabular}
\end{center}

\caption[Cluster 16 top process GO terms]{\sf \textbf{Cluster 16 top process GO terms}}
\label{tab:cls16-process}
\end{table}
Table \ref{tab:cls16-process}
% Booktabs require to add \usepackage{booktabs} to your document preamble
\begin{table}[h]
\begin{center} \sf
\begin{tabular}{p{.7\textwidth}r}
\toprule
\textbf{Name}                                                    & \textbf{Total Score (mean)} \\ \midrule
carbonate dehydratase activity                                   & 4264.07                     \\
cysteine-type peptidase activity                                 & 2253.39                     \\
carbon-nitrogen ligase activity, with glutamine as amido-N-donor & 1652.90                     \\
Rho guanyl-nucleotide exchange factor activity                   & 1446.25                     \\
zinc ion binding                                                 & 1412.73                     \\
guanyl-nucleotide exchange factor activity                       & 1208.72                     \\
nucleic acid binding                                             & 810.14                      \\
peptidase activity                                               & 715.27                      \\
beta-tubulin binding                                             & 707.48                      \\
hydrolase activity, acting on glycosyl bonds                     & 652.43                      \\
glutaminyl-tRNA synthase (glutamine-hydrolyzing) activity        & 550.56                      \\
protein binding                                                  & 433.44                      \\
protein serine/threonine kinase activity                         & 413.31                      \\
metal ion binding                                                & 408.70                      \\
microtubule binding                                              & 403.62                      \\
ATP binding                                                      & 395.77                      \\
transferase activity                                             & 387.71                      \\
maltose alpha-glucosidase activity                               & 378.58                      \\
hydrolase activity                                               & 366.71                      \\
catalytic activity                                               & 331.87                      \\
ligase activity                                                  & 310.07                      \\
alpha-glucosidase activity                                       & 285.38                      \\
nucleotide binding                                               & 239.99                      \\
lyase activity                                                   & 212.94                      \\
oxidoreductase activity                                          & 206.49                      \\
protein kinase activity                                          & 200.95                      \\ \bottomrule                   
\end{tabular}
\end{center}

\caption[Cluster 16 top function GO terms]{\sf \textbf{Cluster 16 top function GO terms}}
\label{tab:cls16-function}
\end{table}
Table \ref{tab:cls16-function}
% % Booktabs require to add \usepackage{booktabs} to your document preamble
\begin{table}[h]
\begin{center} \sf
\begin{tabular}{@{}lr@{}}
\toprule
\textbf{Name}                      & \textbf{Total Score (mean)} \\ \midrule
lysosome                           & 3960.21                     \\
nucleus                            & 1277.31                     \\
integral to membrane               & 943.86                      \\
nucleolus                          & 781.11                      \\
membrane                           & 615.70                      \\
SCF ubiquitin ligase complex       & 572.11                      \\
Cul4-RING ubiquitin ligase complex & 563.83                      \\
plasma membrane                    & 527.87                      \\
cytoplasm                          & 488.72                      \\
intracellular                      & 484.82                      \\
cytoskeleton                       & 305.53                      \\ \bottomrule
\end{tabular}
\end{center}

\caption[Cluster 16 top cellular GO terms]{\sf \textbf{Cluster 16 top cellular GO terms}}
\label{tab:cls16-cellular}
\end{table}
% Table \ref{tab:cls16-cellular}

% cls22
% Booktabs require to add \usepackage{booktabs} to your document preamble
\begin{table}[hp]
\begin{center} \sf
\begin{tabular}{p{.7\textwidth}r}
\toprule
\textbf{Name}                                  & \textbf{Total Score (mean)} \\ \midrule
lysyl-tRNA aminoacylation                      & 5305.29                     \\
regulation of translational initiation         & 5198.13                     \\
protein folding                                & 5145.48                     \\
translation                                    & 3885.25                     \\
tRNA aminoacylation for protein translation    & 3277.15                     \\
translational initiation                       & 3268.53                     \\
cellular protein metabolic process             & 2591.65                     \\
regulation of transcription, DNA-dependent     & 2500.50                     \\
transmembrane transport                        & 2120.54                     \\
formation of translation preinitiation complex & 1607.78                     \\
transcription, DNA-dependent                   & 1314.06                     \\
signal transduction                            & 853.12                      \\
regulation of insulin secretion                & 709.97                      \\
negative regulation of BMP signaling pathway   & 526.20                      \\
regulation of cell cycle                       & 470.82                      \\
melanocyte differentiation                     & 316.77                      \\
endocrine pancreas development                 & 296.05                      \\
transport                                      & 288.59                      \\
developmental pigmentation                     & 287.51                      \\
pigmentation                                   & 273.58                      \\
tRNA processing                                & 223.64                      \\
multicellular organismal development           & 203.63                      \\ \bottomrule
\end{tabular}
\end{center}

\caption[Cluster 22 top process GO terms]{\sf \textbf{Cluster 22 top process GO terms}}
\label{tab:cls22-process}
\end{table}
Table \ref{tab:cls22-process}
% Booktabs require to add \usepackage{booktabs} to your document preamble
\begin{table}[hp]
\begin{center} \sf
\begin{tabular}{p{.7\textwidth}r}
\toprule
\textbf{Name}                               & \textbf{Total Score (mean)} \\ \midrule
translation initiation factor activity      & 20221.31                    \\
unfolded protein binding                    & 7160.14                     \\
DNA binding                                 & 4535.66                     \\
arsenite transmembrane transporter activity & 1600.53                     \\
lysine-tRNA ligase activity                 & 1557.61                     \\
cytoskeletal adaptor activity               & 1456.65                     \\
ATP binding                                 & 1146.99                     \\
aminoacyl-tRNA ligase activity              & 682.47                      \\
nucleotide binding                          & 669.32                      \\
ligase activity                             & 526.89                      \\
transporter activity                        & 400.96                      \\
zinc ion binding                            & 257.13                      \\ \bottomrule                     
\end{tabular}
\end{center}

\caption[Cluster 22 (up at 4h) mean function GO terms]{\sf \textbf{Cluster 22 (up at 4h) mean function GO terms}}
\label{tab:cls22-function}
\end{table}
Table \ref{tab:cls22-function}
% % Booktabs require to add \usepackage{booktabs} to your document preamble
\begin{table}[h]
\begin{center} \sf
\begin{tabular}{@{}lr@{}}
\toprule
\textbf{Name}                                      & \textbf{Total Score (mean)} \\ \midrule
eukaryotic translation initiation factor 3 complex & 9146.56                     \\
integral to nuclear inner membrane                 & 6268.74                     \\
chaperonin-containing T-complex                    & 4904.34                     \\
cytoplasm                                          & 3337.48                     \\
eukaryotic 43S preinitiation complex               & 3196.30                     \\
eukaryotic 48S preinitiation complex               & 3129.90                     \\
nucleus                                            & 1534.04                     \\
lysosomal membrane                                 & 1133.53                     \\
integral to membrane                               & 996.11                      \\
cytosol                                            & 520.36                      \\
nuclear envelope                                   & 469.02                      \\
membrane                                           & 281.10                      \\
nuclear inner membrane                             & 219.28                      \\
nucleolus                                          & 203.40                      \\ \bottomrule
\end{tabular}
\end{center}

\caption[Cluster 22 top cellular GO terms]{\sf \textbf{Cluster 22 top cellular GO terms}}
\label{tab:cls22-cellular}
\end{table}
% Table \ref{tab:cls22-cellular}

% 
% \input{/home/gus/Dropbox/common/projects/Aa_Ag_Cq_As/gfunc_stuff/prelim_gene_analysis/ecr_OR_insect_20130903/ptci_1_0/clusters/cls7/pandas_out/process.tex}
% Table \ref{tab:cluster7-P-mean-TS}
% 
% 




%%% Local Variables: ***
%%% mode: latex ***
%%% TeX-master: "thesis.tex" ***
%%% End: ***
