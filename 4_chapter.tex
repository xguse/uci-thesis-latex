

\chapter{Results} \label{chap:4}

\section{Preliminary data preparation}

\section{gFunc-based analysis}

\subsection{PTCI calculations}

\gls{PTCI} data was calculated using two complementary sets of \gls{TFBS} models.
One set focuses the \gls{PTCI} on genes that have been associated with \gls{20E} and its nuclear receptor (\PTCIe), and a second provides a general focus on \gls{TFBS} models provided by JASPAR that are defined in insects at large (\PTCIi).


\begin{figure}[hp]
%
\subcaptionbox{\label{fig:ecr-pair-ptci-hists-base}}
{\includegraphics[width=.5\linewidth]{figures/figs/ecr_team_ptci_20130918_orthodb7/pairwise_ptci_hist.pdf}}
% 
\subcaptionbox{\label{fig:ecr-pair-ptci-hists-rcum-hist}}
{\includegraphics[width=.5\linewidth]{figures/figs/ecr_team_ptci_20130918_orthodb7/pairwise_ptci_cum_hist.pdf}}
% 
\subcaptionbox{\label{fig:ecr-pair-ptci-hists-fdr}}
{\includegraphics[width=.5\linewidth]{figures/figs/ecr_team_ptci_20130918_orthodb7/pairwise_ptci_fdr.pdf}}
% 
% 
\caption[Pairwise 20E-PTCI results]{\sf \textbf{Pairwise PTCI results for the \gls{20E}-\gls{TFBS} group}:\\
\textbf{(A)} Histogram of mean PTCI results vs null distributions.
\textbf{(B)} Reverse Cumulative histogram of mean PTCI results vs null distributions.
\textbf{(C)} False discovery rate vs PTCI threshold.}
\label{fig:ecr-pair-ptci-hists}
\end{figure}
EcR-pair (Figure \ref{fig:ecr-pair-ptci-hists})


\begin{figure}[hp]
%
\subcaptionbox{\label{fig:ecr-mean-ptci-hists-base}}
{\includegraphics[width=.5\linewidth]{figures/figs/ecr_team_ptci_20130918_orthodb7/mean_ptci_hist.pdf}}
% 
\subcaptionbox{\label{fig:ecr-mean-ptci-hists-rcum-hist}}
{\includegraphics[width=.5\linewidth]{figures/figs/ecr_team_ptci_20130918_orthodb7/mean_ptci_cum_hist.pdf}}
% 
\subcaptionbox{\label{fig:ecr-mean-ptci-hists-fdr}}
{\includegraphics[width=.5\linewidth]{figures/figs/ecr_team_ptci_20130918_orthodb7/mean_ptci_fdr.pdf}}
% 
% 
\caption[Mean 20E-PTCI results]{\sf \textbf{Mean PTCI results for the \gls{20E}-\gls{TFBS} group}:\\
\textbf{(A)} Histogram of mean PTCI results vs null distributions.
\textbf{(B)} Reverse Cumulative histogram of mean PTCI results vs null distributions.
\textbf{(C)} False discover rate vs PTCI threshold.}
\label{fig:ecr-mean-ptci-hists}
\end{figure}
EcR-mean (Figure \ref{fig:ecr-mean-ptci-hists})


\begin{figure}[hp]
%
\subcaptionbox{\label{fig:insect-pair-ptci-hists-base}}
{\includegraphics[width=.5\linewidth]{figures/figs/jaspar_insect_ptci_20130918_orthodb7/pairwise_ptci_hist.pdf}}
% 
\subcaptionbox{\label{fig:insect-pair-ptci-hists-rcum-hist}}
{\includegraphics[width=.5\linewidth]{figures/figs/jaspar_insect_ptci_20130918_orthodb7/pairwise_ptci_cum_hist.pdf}}
% 
\subcaptionbox{\label{fig:insect-pair-ptci-hists-fdr}}
{\includegraphics[width=.5\linewidth]{figures/figs/jaspar_insect_ptci_20130918_orthodb7/pairwise_ptci_fdr.pdf}}
% 
% 
\caption[Pairwise insect-PTCI results]{\sf \textbf{Pairwise PTCI results for the insect-\gls{TFBS} group}:\\
\textbf{(A)} Histogram of mean PTCI results vs null distributions.
\textbf{(B)} Reverse Cumulative histogram of mean PTCI results vs null distributions.
\textbf{(C)} False discover rate vs PTCI threshold.}
\label{fig:insect-pair-ptci-hists}
\end{figure}
Insect-pair (Figure \ref{fig:insect-pair-ptci-hists})


\begin{figure}[hp]
% /home/gus/Dropbox/repos/git/uci-thesis-latex/figures/figs/jaspar_insect_ptci_20130918_orthodb7/mean_ptci_cum_hist.pdf
\subcaptionbox{\label{fig:insect-mean-ptci-hists-base}}
{\includegraphics[width=.5\linewidth]{figures/figs/jaspar_insect_ptci_20130918_orthodb7/mean_ptci_hist.pdf}}
% 
\subcaptionbox{\label{fig:insect-mean-ptci-hists-rcum-hist}}
{\includegraphics[width=.5\linewidth]{figures/figs/jaspar_insect_ptci_20130918_orthodb7/mean_ptci_cum_hist.pdf}}
% 
\subcaptionbox{\label{fig:insect-mean-ptci-hists-fdr}}
{\includegraphics[width=.5\linewidth]{figures/figs/jaspar_insect_ptci_20130918_orthodb7/mean_ptci_fdr.pdf}}
% 
% 
\caption[Mean insect-PTCI results]{\sf \textbf{Mean PTCI results for the insect-\gls{TFBS} group}:\\
\textbf{(A)} Histogram of mean PTCI results vs null distributions.
\textbf{(B)} Reverse Cumulative histogram of mean PTCI results vs null distributions.
\textbf{(C)} False discovery rate vs PTCI threshold.}
\label{fig:insect-mean-ptci-hists}
\end{figure}
Insect-mean (Figure \ref{fig:insect-mean-ptci-hists})




\section{Characterization of Results}


\begin{landscape}

    \begin{figure}[h]
    \centering
    \includegraphics[width=\linewidth]{figures/figs/ecr_and_insects_ptci_20130918_orthodb7/23clusters_ptci_0_95_orthodb7.pdf}
    \caption[\Ag\ clustered abundance profiles]{\sf \textbf{\Ag\ clustered abundance profiles.} \\ 
    Clusters were generated using k-means clustering as implemented in Biopython version 1.62 \cite{Cock2009}.  Abundance profile data was log transformed after one FPKM was added to all data to remove zeros ($log_{10}(\mathrm{FPKM}+1)$).  K-means was then applied using the arithmetic mean as the center for cluster definition.  The data displayed here is the gene-wise standardization of the \textbf{raw} FPKM data such that each profile has mean = 0 and standard deviation = 1. Each median abundance profile is marked by a thick gray line. \textbf{Panel Titles} - ID: cluster identifier | members: number of genes in cluster | percentage of genes represented in all clusters. \textbf{Time Points} - \gls{NBF}, 4, 6, 8, 10 h \gls{PBM}.
}
    \label{fig:23-clusters}
    \end{figure}
    
\end{landscape}


Figure \ref{fig:23-clusters}

% 
\begin{figure}[hp]
% 
\begin{subfigure}[t]{.5\linewidth}
\includegraphics[width=\linewidth]{figures/figs/ecr_and_insects_ptci_20130903/upAt4_gene_profiles_from_cummerbund/Aa_upAt4_cls6_Ag_target_FPKMs_vb_orthos.pdf}
\caption{}
\label{fig:cluster6-Aa}
\end{subfigure}%
%
\begin{subfigure}[t]{.5\linewidth}
\includegraphics[width=\linewidth]{figures/figs/ecr_and_insects_ptci_20130903/upAt4_gene_profiles_from_cummerbund/Ag_upAt4_cls6_Ag_target_FPKMs_vb_orthos.pdf}
\caption{}
\label{fig:cluster6-Ag}
\end{subfigure}
% 
\begin{subfigure}[t]{.5\linewidth}
\includegraphics[width=\linewidth]{figures/figs/ecr_and_insects_ptci_20130903/upAt4_gene_profiles_from_cummerbund/Cq_upAt4_cls6_Ag_target_FPKMs_vb_orthos.pdf}
\caption{}
\label{fig:cluster6-Cq}
\end{subfigure}
% 
\caption[Orthologs of cluster 6]{\sf \textbf{Orthologs of cluster 6:}\\

\textbf{(A)} \Aa.
\textbf{(B)} \Ag.
\textbf{(C)} \Cq.
\todo[inline,caption={Finish Fig \ref{fig:cluster6}}]{ 
\begin{itemize}
    \item the text
    \item double gene names
    \item fix fig widths
\end{itemize}}
}
\label{fig:cluster6}
\end{figure}
% Figure \ref{fig:cluster6}
% 
% 
\begin{figure}[hp]
% 
\begin{subfigure}[t]{.5\linewidth}
\includegraphics[width=\linewidth]{figures/figs/ecr_and_insects_ptci_20130903/upAfter4_gene_profiles_from_cummerbund/Aa_upAfter4_cls7_Ag_target_FPKMs_vb_orthos.pdf}
\caption{}
\label{fig:cluster7-Aa}
\end{subfigure}%
%
\begin{subfigure}[t]{.5\linewidth}
\includegraphics[width=\linewidth]{figures/figs/ecr_and_insects_ptci_20130903/upAfter4_gene_profiles_from_cummerbund/Ag_upAfter4_cls7_Ag_target_FPKMs_vb_orthos.pdf}
\caption{}
\label{fig:cluster7-Ag}
\end{subfigure}
% 
\begin{subfigure}[t]{.5\linewidth}
\includegraphics[width=\linewidth]{figures/figs/ecr_and_insects_ptci_20130903/upAfter4_gene_profiles_from_cummerbund/Cq_upAfter4_cls7_Ag_target_FPKMs_vb_orthos.pdf}
\caption{}
\label{fig:cluster7-Cq}
\end{subfigure}
% 
\caption[Orthologs of cluster 7]{\sf \textbf{Orthologs of cluster 7:}\\

\textbf{(A)} \Aa.
\textbf{(B)} \Ag.
\textbf{(C)} \Cq.
\todo[inline,caption={Finish Fig \ref{fig:cluster7}}]{ 
\begin{itemize}
    \item the text
    \item double gene names
    \item fix fig widths
\end{itemize}}
}
\label{fig:cluster7}
\end{figure}
% Figure \ref{fig:cluster7}
% 
% \input{/home/gus/Dropbox/common/projects/Aa_Ag_Cq_As/gfunc_stuff/prelim_gene_analysis/ecr_OR_insect_20130903/ptci_1_0/clusters/cls7/pandas_out/process.tex}
% Table \ref{tab:cluster7-P-mean-TS}
% 
% 
% 
\begin{figure}[hp]
% 
\begin{subfigure}[t]{.5\linewidth}
\includegraphics[width=\linewidth]{figures/figs/ecr_and_insects_ptci_20130903/downAt4_gene_profiles_from_cummerbund/Aa_downAt4_cls19_Ag_target_FPKMs_vb_orthos.pdf}
\caption{}
\label{fig:cluster19-Aa}
\end{subfigure}%
%
\begin{subfigure}[t]{.5\linewidth}
\includegraphics[width=\linewidth]{figures/figs/ecr_and_insects_ptci_20130903/downAt4_gene_profiles_from_cummerbund/Ag_downAt4_cls19_Ag_target_FPKMs_vb_orthos.pdf}
\caption{}
\label{fig:cluster19-Ag}
\end{subfigure}
% 
\begin{subfigure}[t]{.5\linewidth}
\includegraphics[width=\linewidth]{figures/figs/ecr_and_insects_ptci_20130903/downAt4_gene_profiles_from_cummerbund/Cq_downAt4_cls19_Ag_target_FPKMs_vb_orthos.pdf}
\caption{}
\label{fig:cluster19-Cq}
\end{subfigure}
% 
\caption[Orthologs of cluster 19]{\sf \textbf{Orthologs of cluster 19:}\\

\textbf{(A)} \Aa.
\textbf{(B)} \Ag.
\textbf{(C)} \Cq.
\todo[inline,caption={Finish Fig \ref{fig:cluster19}}]{Finish this fig: 
\begin{itemize}
    \item the text
    \item double gene names
    \item fix fig widths
\end{itemize}}
}
\label{fig:cluster19}
\end{figure}
% Figure \ref{fig:cluster19}



%%% Local Variables: ***
%%% mode: latex ***
%%% TeX-master: "thesis.tex" ***
%%% End: ***
